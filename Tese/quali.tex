\documentclass[12pt,a4paper]{article}
\usepackage[margin= 3cm]{geometry}
\usepackage[utf8]{inputenc}
%\usepackage[russian]{babel}
\usepackage[T1]{fontenc}  % T1 é o normal, T2A é p usar alfabeto cyrilico
\usepackage[portuguese]{babel} % deixar as coisas em portugues
\usepackage{amsmath}
\usepackage{amsfonts}
\usepackage{amssymb}
\usepackage{graphicx}
\usepackage{subcaption}
\usepackage{setspace}	% espaçamento do texto
\usepackage{multirow}
\usepackage{tabularx}	% upgrade das tabelas
\usepackage{hyperref}  % coiso pras ref. ficar todas linkadinhas        
\hypersetup{colorlinks, citecolor=blue, linkcolor=blue, urlcolor=blue}
\usepackage[toc,page]{appendix}
\usepackage{pdfpages}
\usepackage{comment}
\usepackage{svg}  
\usepackage{float} % Usado para posicionamento de imagens
\usepackage{pgf}



\begin{document}


\begin{titlepage}
    \begin{center}
        \setstretch{2.0}
        \begin{tabular}{m{4cm}m{10cm}}
            \multirow{3}{*}{\vspace{-0.5cm}\includegraphics[width=3cm]{if.png}}
            & {\LARGE Universidade de São Paulo} \\
            & {\LARGE Instituto de física} \\
            & {\LARGE Física aplicada} \\
        \end{tabular}
        \vspace{1.8cm}
    
        {\LARGE Qualificação de Doutorado}
        \vspace{4.0cm}

        {\LARGE {\bf Transporte de partículas por ondas eletrostáticas}}
        \vfill

        \setstretch{1}    
        {\Large Acadêmico: André Farinha Bósio}
        \vspace{0.8cm}
    
        {\Large Orientador: Prof. Dr. Iberê Luiz Caldas}
        \vspace{0.8cm}
        
        {\Large Coorientador: Prof. Dr. Ricardo Luiz Viana}
        \vspace{0.8cm}
        
        
        
        \vspace{1.2cm}
        
        {\large São Paulo, \today}
    \end{center}
\end{titlepage}


\begin{titlepage}
    \begin{center}
        \setstretch{2.0}
        \begin{tabular}{m{4cm}m{10cm}}
            \multirow{3}{*}{\vspace{-0.5cm}\includegraphics[width=3cm]{if.png}}
            & {\LARGE Universidade de São Paulo} \\
            & {\LARGE Instituto de física} \\
            & {\LARGE Física aplicada} \\
        \end{tabular}
        \vspace{1.8cm}
    
        {\LARGE Qualificação de Doutorado}
        \vspace{4.0cm}

        {\LARGE {\bf Transporte de partículas por ondas eletrostáticas}}
        \vfill

        \setstretch{1}    
        \hspace*{7.0cm}\parbox{9.0cm}
        {\large Tese de doutorado submetida à CPG, sob orientação do Prof. Dr. Iberê Luiz Caldas como parte dos requisitos para obtenção do título de doutor em física.}
        \vfill
    
         {\Large Acadêmico: André Farinha Bósio}
        \vspace{0.8cm}
    
        {\Large Orientador: Prof. Dr. Iberê Luiz Caldas}
        \vspace{0.8cm}
        
        {\Large Coorientador: Prof. Dr. Ricardo Luiz Viana}
        \vspace{0.8cm}
        
        
        \vspace{1.2cm}
        
        {\large São Paulo, \today}
    
    \end{center}
\end{titlepage}


\clearpage
\tableofcontents

\setstretch{1.5}
\clearpage

\section{Hamiltoniano do modelo} %%%%%%%%%%%%%%%%%%%%%%%%%%%%%%%%%

Nosso modelo 

Nosso sistema consiste de uma regiao quadrada do espaço, com lados com comprimento $2\pi$, com condiçao periodica de contorno
Assumindo agora um campo magnético na direção $\vec{z}$.

\begin{equation}
\vec{B} = B_0 \vec{z}
\end{equation}

\noindent bem como um campo elétrico radial da forma 
\begin{equation}
\vec{E} = - \nabla \phi(x,y,t)    
\end{equation}

\noindent de tal forma que $\vec{B} \perp \vec{E}$, gerando assim uma velocidade de deriva elétrica

\begin{equation}
\vec{v}_E = \frac{\vec{E} \times \vec{B}}{B^2}    
\label{eqderiva}
\end{equation}

Desenvolvendo a equação \ref{eqderiva} com nosso campos chegamos em

\begin{equation}
v_x = \frac{dx}{dt} = -\frac{1}{B_0}\frac{\partial}{\partial y} \phi(x,y,t) \hspace{1.5cm} v_y = \frac{dy}{dt} = \frac{1}{B_0}\frac{\partial}{\partial x} \phi(x,y,t)
\end{equation}

\noindent e comparando agora com as equações de Hamilton

\begin{equation}
\frac{dx}{dt} = -\frac{\partial}{\partial y} H(x,y,t) \hspace{1.5cm}  \frac{dy}{dt} = \frac{\partial}{\partial x} H(x,y,t)
\label{eqvs}
\end{equation}

\noindent vemos que o hamiltoniano do sistema  está ligada ao potencial elétrico pela relação

\begin{equation}
H(x,y,z) = \frac{\phi(x,y,z)}{B_0}
\end{equation}

\noindent de maneira que $x$ e $y$ formam um par momento coordenada, com x fazendo o papel de momento e y de coordenada.
Novamente, seguindo o modelo de Horton \cite{horton1985onset}, o potencial usado será uma soma de ondas eletrostáticas, se propagando na direção poloidal e estacionárias na direção radial, temos ainda  um potencial de equilíbrio $\phi_0(x)$

\begin{equation}
\phi(x,y,t) = \phi_0(x) + \sum_i A_i \sen(k_{xi}x)sin(k_{yi}y - \omega_i t)
\label{Potencial}\
\end{equation}

Dessa forma nosso hamiltoniano toma a forma

\begin{equation}
H(x,y,t) = \frac{\phi_0(x)}{B_0} + \sum_i \frac{A_i}{B_0} \sen(k_{xi}x)sin(k_{yi}y - \omega_i t)
\label{Potencial}
\end{equation}

\subsection*{Uma onda}

Quando o sistema possui apenas uma onda o hamiltoniano se reduz a

\begin{equation}
H(x,y,t) = \frac{\phi_0(x)}{B_0} +\frac{A}{B_0} \sen(k_{x}x)\sen(k_{y}y - \omega_i t)
\label{Potencial}
\end{equation}

\noindent por conveniencia vamos mudar o referencial do sistema para um que se move com mesma velocidade de fase da onda. Faremos isso usando uma transformaçao canônica utilizando a função

\begin{equation}

F_2 = 

\end{equation}

\section{Sistema com duas ondas}

Os valores usados para as simulações são:


\begin{tabular}{c|cccc}
\hline 
i & $A_i$ & $v$ & $\k_y$ & $\k_x$ \\ 
\hline 
0 & 1 & 1 & 3 & 3 \\ 
\hline 
1 & - & 2 & 3 & 3 \\ 
\hline 
\end{tabular} 

Com a $\Delta v$ entre as ondas como 1 por simplicidade, dessa forma os $\omega_i$ podem ficar livres, ja que temos o elo entre $\Delta v$ e as frequencias.

\begin{figure}
\includegraphics{data/A2_D_gamma_phase.pdf}
\caption{Expoente do deslocamento quadratico medio ao longo do tempo, vemos que algumas combinacoes geram transporte altamente anomalo}
\end{figure} 







\clearpage
\bibliographystyle{ieeetr}
\bibliography{ref}





\end{document}